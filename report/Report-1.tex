\documentclass[12pt]{article}

% Packages
\usepackage{graphicx} % For including images
\usepackage{geometry} % For adjusting page margins
\usepackage{titlesec} % For customizing section titles
\usepackage{fancyhdr} % For customizing headers and footers
\usepackage{hyperref} % For adding hyperlinks

% Page setup
\geometry{a4paper, margin=1in}
\pagestyle{fancy}
\fancyhf{}
\fancyhead[L]{\leftmark}
\fancyhead[R]{RTGP Project Report}
\fancyfoot[C]{\thepage}

% Section title formatting
\titleformat{\section}
{\normalfont\Large\bfseries}{\thesection}{1em}{}
\titleformat{\subsection}
{\normalfont\large\bfseries}{\thesubsection}{1em}{}

% Begin document
\begin{document}

% Title
\begin{titlepage}
    \centering
    \vspace*{2cm}
    {\Huge\bfseries Real Time Graphics Programming Project Report\par}
    \vspace{1.5cm}
    {\Large\itshape Compagnoni Alessandro\par}
    \vspace{0.5cm}
   {\large UNIMI A.A. 2023/2024\par}
    \vfill
    \includegraphics[width=0.8\textwidth]{Images/logoUnimi.png} % Replace with your logo/image
    \vfill
\end{titlepage}

% Table of Contents
\tableofcontents
\newpage

% Sections
\section{Overview}
\label{sec:overview}

This project implements a 3D environment using OpenGL, allowing users to navigate within 3 square rooms connected by corridors. Inside the rooms various 3D objects are placed, each rendered with a different material based on shaders that showcase various types of noise with different parameters, all freely adjustable by the user thanks to the dedicated UI.

\section{Project Details and Design Choices}
\label{sec:project_details}
\subsection{Technologies}
The following technologies were selected to ensure efficient rendering, cross-platform compatibility, and streamlined development for the project:

\begin{itemize}
\item OpenGL Modern Pipeline (Version 3.3+):
\newline
OpenGL’s modern shader-based architecture was chosen to leverage direct control over vertex and fragment processing. By utilizing GLSL shaders, the pipeline enables complex procedural effects while maintaining high performance.

\item GLFW (Graphics Library Framework):
\newline
GLFW provides robust window management and input handling, ensuring consistent behavior across operating systems, also its event-driven architecture simplifies interaction with user inputs.

\item GLM (OpenGL Mathematics):
\newline
GLM mathematics library is optimized for graphics programming. It offers essential data types (e.g., vectors, matrices) and prebuilt functions for common transformations (e.g., translation, projection).
\end{itemize}

\subsection{Architecture}

\newpage

\section{Algorithms and Techniques}
\label{sec:algorithms}

\newpage

\section{Implementation Details}
\label{sec:implementation}

\newpage

\section{Performance Evaluation}
\label{sec:performance}

\newpage

\section{Conclusion}
\label{sec:conclusion}

% End document
\end{document}